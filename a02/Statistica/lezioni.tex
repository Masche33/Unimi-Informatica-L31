\documentclass{article}
\usepackage[italian]{babel}
\usepackage[utf8]{inputenc}
\usepackage{amsmath}
\usepackage{tcolorbox}
\usepackage{amsfonts}
\usepackage{geometry}
\usepackage{pgfplots}
\usepackage{caption}
\usepackage{pgf-pie}
\pgfplotsset{compat=1.16}

\title{Statistica \& Analisi dei dati}
\author{Mascherpa Matteo}
\date{a.a. 2024-2025}

\begin{document}

\maketitle

\tableofcontents

\section{Introduzione alla Statistica}
  \subsection{Definizioni:}
    \begin{description}
      \item[Statistica:] La \textit{statistica} è l'arte di apprendere dai dati. La statistica si occupa della raccolta, della descrizione e dell'analisi dei dati, possibilmente permettendo di trarne delle conclusioni.
      \item[Statistica descrittiva:] La parte della statistica che si occupa di descrivere e riassumere i dati.
      \item[Statistica inferenziale:] La parte della statistica che si occupa di trarre conclusioni dai dati. (Richiede quindi di conoscere i concetti di probabilità)
      \item[Popolazione:] L'insieme della totalità dei dati.
      \item[Campione:] Sottoinsieme omogeneo delle popolazione. 
      \item[Campione Casuale:] Il campione è un sottoinsieme della popolazione dove ogni elemento ha la stessa probabilità di essere scelto.
      \item[Frequenza assoluta:] Numero di volte che un dato compare in un campione. Indicata con $f_j$ dove $j$ è un elemento del campione.
      \item[Frequenza relativa:] Frazione di volte che un dato compare nel campione. Indicata con $f'_j (= \frac{f_j}{n})$.
    \end{description}

  \subsection{Popolazione e Campioni}
    In statistica si vuole ottenere informazioni da un insieme di dati(\textit{popolazione}. Ma dato la natura delle popolazione, spesso enorme, se ne prende un sottoinsieme di distribuzione omogenea(\textit{Campione}) dove ogni elemento ha una stessa probabilità di essere scelto. In caso il campione non abbia questa qualità le informazioni da esso estratto sono inconcludenti poiché non \textit{rappresentativo}. Una volta ottenuto il \textit{campione casuale} si può usare tecniche di \textit{statistica inferenziale} per ottenere da esso le informazioni volute(dato che dal campione si possano ricavare, da un insieme di dati sul colore dei vestiti non si potrà in ogni caso trovare alcun dato sui gusti dei gelati). 
  
    \subsubsection*{Campione casuale stratificato}
      Prendere valori a caso non è spesso la migliore delle scelte, è una scelta buona, ma esistono opzioni migliori. Per questo esiste il concetto di \textbf{\textit{campione casuale stratificato}}. Prima di prendere gli elementi casuali del campione si divide la popolazione in \textit{classi} secondo un qualche criterio. Una volta si prende da ogni classe un numero di elementi proporzionale alla cardianlità della classe. Se una classe è composta dal $15\%$ della popolazione allora il $15\%$ del campione dovrà essere preso casualmente dalla quella classe, e così via. 

\section{Rappresentazione dei dati}
  Serve un metodo per rendere i dati leggibili e facilmente interpretabili, a questo scopo ci vengono in soccorso un potente strumento grafico, i \textit{grafici}.

  \subsection{Tabelle e Grafici}
    Avendo un insieme di dati di dati disposti nel modo più elementare possibile la lettura diventa complessa, invece, gli stessi dati, una volta riarrangiati vengono letti con semplicità.
  
    \begin{minipage}[t]{0.45\textwidth}
    \begin{flushleft}
      2, 2, 0, 0, 5, 8, 3, 4, 1, 0, 0, 7, 1, 7, 1, 5, 4, 0, 4, 0, 1, 8, 9, 7, 0, 1, 7, 2, 5, 5, 4, 3, 3, 0, 0, 2, 5, 1, 3, 0, 1, 0, 2, 4, 5, 0, 5, 7, 5, 1
    \end{flushleft}
    \end{minipage}%
    \hfill
    \begin{minipage}[t]{0.45\textwidth}
    \begin{tabular}{|c|c|}
      \hline
      \textbf{Valore} & \textbf{Frequenza} \\
      \hline
      0 & 12 \\
      \hline
      1 & 8 \\
      \hline
      2 & 5 \\
      \hline
      3 & 4 \\
      \hline
      4 & 5 \\
      \hline
      5 & 8 \\
      \hline
      7 & 6 \\
      \hline
      8 & 2 \\
      \hline
      9 & 1 \\
      \hline
    \end{tabular}
    \end{minipage}

  \subsection{Grafici a bastoncini, a barre \& poligonali}
    Varie sono le tipologie di grafici, tra le opzioni ci sono grafici a bastoncini, a barre e poligonali.
  
    \begin{figure}[h]
      \begin{minipage}{0.35\textwidth} % Grafico occupa il 35% della larghezza
        DRAW HERE A STICK CHART. 
      \end{minipage}
      \hfill
      \begin{minipage}{0.6\textwidth} % Testo occupa il 60% della larghezza
        Grafico dalla lettura intuitiva dove ogni dato è rappresentato da un semplice segmento.
      \end{minipage}
    \end{figure}
    \begin{figure}[h]
      \begin{minipage}{0.35\textwidth} % Grafico occupa il 35% della larghezza
        \begin{tikzpicture}
          \begin{axis}[
              ybar,
              ymin=0, ymax=13,
              symbolic x coords={0, 1, 2, 3, 4, 5, 6, 7, 8, 9},
              xtick=data,
              axis lines=left,
              enlarge x limits=0.1,
              enlarge y limits={upper, value=0.2},
              width=\textwidth, % Larghezza del grafico uguale alla larghezza del testo
              bar width=8pt, % Larghezza delle barre
              legend style={at={(0.5,-0.2)}, anchor=north, legend columns=-1}
          ]
              % Primo set di dati
              \addplot[black, fill=black!80] coordinates {
                  (0, 12)
                  (1, 8)
                  (2, 5)
                  (3, 4)
                  (4, 5)
                  (5, 8)
                  (6, 0)
                  (7, 5)
                  (8, 2)
                  (9, 1)
              };
  
          \end{axis}
        \end{tikzpicture}
      \end{minipage}
      \hfill
      \begin{minipage}{0.6\textwidth} % Testo occupa il 60% della larghezza
        Spesso utilizzati per la rappresentazione di dati, per ogni dato sull'asse delle ascisse esiste un parallepipedo dall'altezza equivalente al valore del dato che si vuole sappresentare. Comodo per la visualizzazione dei valori di varie categorie diverse in un unico grafico.
      \end{minipage}
    \end{figure}
    \begin{figure}[h]
      \begin{minipage}{0.35\textwidth} % Grafico occupa il 35% della larghezza
        \begin{tikzpicture}
          \begin{axis}[
              axis lines=left,
              xmin=-0.5, xmax=9.5,
              ymin=0, ymax=13,
              xtick={0, 1, 2, 3, 4, 5, 6, 7, 8, 9},
              ytick={0, 1, 2, 3, 4, 5, 6, 7, 8, 9, 10, 11, 12},
              width=\textwidth, % Larghezza del grafico uguale alla larghezza del testo
              height=5cm, % Altezza del grafico
              mark=*, % Stile dei marker
              smooth % Linee lisce tra i punti
          ]
              \addplot coordinates {
                  (0, 12)
                  (1, 8)
                  (2, 5)
                  (3, 4)
                  (4, 5)
                  (5, 8)
                  (6, 0)
                  (7, 5)
                  (8, 2)
                  (9, 1)
              };
          \end{axis}
        \end{tikzpicture}
      \end{minipage}
      \hfill
      \begin{minipage}{0.6\textwidth} % Testo occupa il 60% della larghezza
        Il grafico poligonale infine serve a rendere visibile l'andamento di un dato unendo ogni punto con il successivo tramite un segmento creando quandi una sorta di funzione che indica l'andamento del valore sull'asse delle ordinate.    
      \end{minipage}
    \end{figure}
  
  Un insieme di dati si dice simmetrico al valore $x_n$ se le frequenze dei valori $x_n-c$ e $x_n+c$ sono le stesse per ogni $c$. Si dice \textit{quasi simmetrico} se i valori sono precisamente uguali ma sono solamente simili, è quindi una proprietà meno restrittiva.
  
  \subsubsection*{Grafici per le sequenze relative}
  
  A volte è conveniente, al posto di avere le frequenze assolute, visualizzazare le frequenze \textit{relative}. Dato $f$ la frequenza di $x$ allora posso avere un grafico \textit{frequenza relativa} $\frac{f}{n}$ dove $n$ è il numero totale di osservazioni del dato. In caso la somma dei valori delle colonne farà $1$ cioé tutte tutte le osservazioni.
  
  \subsubsection*{Grafici a torta}
  
  \begin{figure}[h]
      \begin{minipage}{0.20\textwidth} % Grafico occupa il 35% della larghezza
       \begin{tikzpicture}
         \pie[radius=1]{10/A, 20/B, 30/C, 40/D}
       \end{tikzpicture}
      \end{minipage}
      \hfill
      \begin{minipage}{0.8\textwidth}
        In caso che i dati non siano numerici una valida opzione è il grafico a torta che indica le frequenze relative di ogni dato. La percentuale di grafico da assegnare ad un solo valore si calcola $\frac{f}{n}$ dove $n$ è il numero totale di osservazioni e per ottenere. Per invece ottenere l'angolo del grafico a torta la formula diventa $\frac{360*f}{n}$.
      \end{minipage}
  \end{figure}
  
  \subsubsection*{Raggrupamenti di dati e istogrammi}
  
   Serve quando la quantità dei dati è tale da rendere inutile la rappresentazione normale in un grafico. In tal caso conviene raggruppare in classi i dati. Trovare il numero perfetto di classi è spesso molto complesso e ci si può accontentare con un compromesso tra, scegliere poche classi(a costo di perdere molte informazioni) e scegliere molte classi(in modo da mantenere il significato dei dati nel campione). Di solito il valore è compreso in $[5,10]$. Il libro usa la convenzione di includere in una classi il suo valore inferiore e non quello superiore: $[\text{lim}_{inf}, \text{lim}_{sup})$. Ogni valore è la media dei valori nella classe.
  
   \subsubsection*{Diagrammi Ramo-Foglia}
  
  
  \hfill
  
  \begin{minipage}[c]{0.30\textwidth}
      \centering
      \fbox{%
        \begin{minipage}{0.91\textwidth}
          \hspace*{-1.5mm}
          \begin{tabular}{r|l}
            22 & 123 \\
            23 & 234, 567, 890 \\
            24 & 345 \\
            25 & 012, 034, 078, 234 \\
            26 & 123, 456, 789 \\
            27 & 234, 345, 456 \\
            28 & 567, 678, 789, 890 \\

          \end{tabular}
        \end{minipage}%
      }
      \vspace{-2mm}
  \end{minipage}
  \hspace{2mm}
  \begin{minipage}[c]{0.6\textwidth}
    Comodo per rapprentare un piccolo campione di dati. Dove il valore a sinistra è il prefisso e i valori a destra sono tutti i valori che hanno quel prefisso. Questo serve a riassumere i dati. In caso un ramo abbia troppe foglie si può dividere su più linee per una maggiore leggibilità.
      \vspace{-3mm}
  \end{minipage}
  \vspace{1mm}
  
\section{Statistica descrittiva}


\end{document}
