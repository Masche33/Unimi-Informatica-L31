\documentclass{article}
\usepackage[italian]{babel}
\usepackage[utf8]{inputenc}
\usepackage{amsmath}
\usepackage{tcolorbox}
\usepackage{amsfonts}
\usepackage{amssymb}
\usepackage{geometry}
\usepackage{pgfplots}
\usepackage{caption}
\pgfplotsset{compat=1.16}

\title{Ricerca Operativa}
\author{Mascherpa Matteo}
\date{a.a. 2024-2025}

\begin{document}

\maketitle

  \section{Introduzione}
    Con lo scalare delle imprese della rivoluzione industriale è diventata preponderante la necessità di allocare meglio le risorse. Spesso se un'operazione cresce velocemente è facile che ogni compartimento diventi un'entità a se stante rendendo complessa la gestione. Da questa necessità è emersa la \textit{Ricerca operativa} che è maturata ed è stata sfruttata per la prima volta nella $WWII$. Servì per allocare nel miglior modo le poche risorse inglesi per evitare di soccombere ai tedeschi. Con l'avvenire dei computer poi quesat materia ha avuto una spinta che la ha portata alla situazione attuale.

  \section{Panoramica all'approccio della modellazione}
    Nonostante la parte più preponderante della materia sia composta da metodi matematici non si deve dimenticare che una parte dello sforzo è la \textit{raccolta dei dati}.
    Un riassunto delle varie fasi di uno studio sono:
    \begin{enumerate}
      \item Definire il problema e ottenere i dati.
      \item Formulo il modello matematico che rappresenti il problema.
      \item Sviluppare la procedura per derivare la soluzione del problema dal modello.
      \item Test, miglioramento, rifinatura fino a quando si è soddisfatti.
      \item Preparare l'applicazione operativa della soluzione.
      \item Implementazione della soluzione trovata.
    \end{enumerate}

    \subsection{Definire il problema}
      Solitamente il problema da risolvere è descritto in termini vaghi, va quindi trovato un modo per formalizzarlo. 
    \subsection{Formulo il modello matematico}

    \subsection{Sviluppare la procedura}

    \subsection{Test, Miglioramento, e rifinatura}

    \subsection{Preparazione dell'applicazione}

    \subsection{Implementazione}
\end{document}
