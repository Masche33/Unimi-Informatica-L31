\documentclass{article}
\usepackage[italian]{babel}
\usepackage[utf8]{inputenc}
\usepackage{amsmath}
\usepackage{tcolorbox}
\usepackage{amsfonts}
\usepackage{amssymb}
\usepackage{geometry}
\usepackage{pgfplots}
\usepackage{caption}
\pgfplotsset{compat=1.16}

\title{Ricerca Operativa}
\author{Mascherpa Matteo}
\date{a.a. 2024-2025}

\begin{document}

\maketitle

  \section{Introduzione}
    Con lo scalare improvviso delle imprese della rivoluzione industriale è diventata preponderante la necessità di allocare al meglio le risorse disponibili. Spesso se un'operazione cresce velocemente è facile che ogni compartimento diventi un'entità a sé stante rendendo complessa la gestione. Da questa necessità è emersa la \textit{Ricerca operativa} che è maturata ed è stata sfruttata per la prima volta nella seconda guerra mondiale. Servì per allocare nel miglior modo le poche risorse inglesi per evitare di soccombere ai tedeschi. Con l'avvenire dei computer poi questa materia ha avuto una spinta che la ha portata alla situazione attuale.

  \section{Panoramica all'approccio della modellazione}
    Nonostante la parte più preponderante della materia sia composta da metodi matematici non si deve dimenticare che una parte dello sforzo è la \textit{raccolta dei dati}.
    Un riassunto delle varie fasi di uno studio sono:
    \begin{enumerate}
      \item Definire il problema e ottenere i dati.
      \item Formulare il modello matematico che rappresenti il problema.
      \item Sviluppare la procedura per derivare la soluzione del problema dal modello.
      \item Test, miglioramento, rifinatura fino a quando si è soddisfatti.
      \item Preparare l'applicazione operativa della soluzione.
      \item Implementazione della soluzione trovata.
    \end{enumerate}

    \subsection{Definire il problema}
      Solitamente il problema da risolvere è descritto in termini vaghi, va quindi trovato un modo per formalizzarlo. Solitamente, non si deve dare la "soluzione" ma un insieme di diverse alternative dove il cliente sceglierà quella che più lo aggrada in base a quale soluzione lo attrae di più(Alcuni potrebbero volere una estrema rapidità altri una maggiore robustezza). Per prima cosa si deve discernere i \textbf{principi fondamentali}, che saranno quelli che guideranno la scelta finale. Generalmente per le imprese i punti salienti sono: \textit{Proprietario, Imppiegati, Clienti, Fornitori, Governo/nazione}. 

    \subsection{Formulo il modello matematico}
      Il modello matematico sono versioni idealizzate della realtà per una più facile lavorazione. Esso è composto da \textbf{\textit{variabili decisionali}}, $(x_1,\dots,x_n)$ che rappresentano i valori come valori determinati. Le misure delle prestazioni con le \textbf{\textit{funzioni obiettivo}} espresse come funzioni aritmetiche, $(P=3x_1+2x_2+\dots+mx_n)$. Ci sono anche modi per avere delle restrizioni, chiamate \textbf{\textit{vincoli}}, espresse come equazioni, $(3x_1+2x_2+\dots+mx_n<10)$. Le costanti del modello vengono chiamate \textbf{\textit{parametro del modello}}. Solitamente, per motivo di incertezza, i parametri non sono altro che grezze stime, concetto a cui ci si riferisce come \textbf{\textit{sensitività d'analisi}}. Una volta modellato il problema esso non diventa altro che un problema di programmazione lineare. Bisogna evitare però di creare un modello insoddisfaciibile essendo il modello astratto e idealizzato ma con un utilizzo concreto. Un criterio per giudicare un modello è la relativa correttezza in base alle scelte prese. Nello sviluppo del modello è buona pratica da cominciare da un nucleo ristretto e arricchirlo, questo processo si chiama \textbf{\textit{arricchimento del modello}}. In caso gli obiettivi fossero molteplici essi vengono trasformati in una misura composita chiamata: \textbf{\textit{Misura globale delle prestazioni}}.

    \subsection{Sviluppare la procedura} 
      Solitamente un passo semplice dove applicare al modello sviluppato precedentemente un \textit{algoritmo standard}, la parte più complessa è l'\textit{analisi postottimale}. Gli algoritmi infatti restituiscono la soluzione \textit{ottime} per il modello dato, il problema è che il modello dato essendo un astrazione può quasi sempre essere migliorato per ottenere una nuova soluzione migliore. \textbf{\textit{Herbert Simon}}, nobel per l'economia, ha definito "\textit{satisficing}" (\textit{soddisfacente}) la tendenza dei manager a cercare una soluzione "\textit{abbastanza buona}" e non per forza ottima. Bisogna, nella modifica del modello porre attenzione alla \textit{sensitività delle analisi}, cioè determinare quale parametro modifica considerevomente la soluzione e la sposta di più verso un obiettivo che più si confà alla realtà del problema.

    \subsection{Test, Miglioramento, e rifinatura}
      Essendo i casi della realtà, al contrario degli esempi didattici, enormi è delirante aspettarsi che la prima soluzione trovata sia quella che alla fine si andrà ad attuare. Per questo esiste questa fase dove il modello viene messo sotto stress fino ad arrivare ad un livello soddisfacente. Non ci si può neanche aspettare di arrivare alla versione perfetta, si punta perciò alla minimizzazione dei bachi. Questo processo si chiama \textbf{\textit{validazione del modello}}. Visto l'ammontare di tempo speso nelle fasi precedenti è facile trascurare, per dimenticanza, vari dettagli. Per questo è spesso consigliato che a questo processo venga aggiunto nel team un elemento totalmente avulso dal progetto. Tra gli errori più frequenti ci sono gli errori di incosistenze nelle unità dei valori, \textit{consistenza dimensionale}. Un approccio sistematico è il \textit{test a retrospettiva}. Ciò consiste nell'utilizzo di dati passati per ricostruire vecchi scenari simili e determinare la risposta del modello. Lo svantaggio è che i dati usati per i test sono gli stessi usati per creare il modello quindi non i migliori per trovare le falle.
      
    \subsection{Preparazione dell'applicazione}
      Una volta testato il modello ed essere arrivati ad una versione da adottare allora si deve creare una documentazione estesa che spieghi ogni dettaglio del modello. Il sistema che verrà creato sarà quasi certamente informatizzato. In questa fase si creano le \textit{basi di dati} e \textit{sistemi di gestione delle informazioni}. In caso le decisioni da prendere siano interattive si crea un \textit{sistema di supporto alle decisioni} in modo da avere un rapido accesso alle possibili migliori decisioni. 

    \subsection{Implementazione}
      La fase critica in cui le scelte prese in precedenza vanno, finalmente, attuate nel modo più accurato possibile al modello ricavato dai mesi di lavoro. Tasselli importanti nell'applicazione sono la comunicazione tra reparti. se tutto viene svolto in cordinazione tra i vari capi reparto il modello potrà essere utilizzato per anni, per questo è importante un riscontro costante per potere avere aggiustamenti minori che vanno documentati e motivati. Grazie a questo si potrà andare avanti fino a quando non cambieranno le necessità che il modello soddisfaceva.
\end{document}
